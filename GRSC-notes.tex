%% ----------------------------------- 
%% ---------- William E. Olsen -------
%% ---------- wolsen@uga.edu ---------
%% -----------------------------------

\documentclass{notes} 

%% ----------------------------------- 
%% ---- Course-Specific Packages -----
%% -----------------------------------
\usepackage{amsthm, amssymb}
\usepackage[left = 1in, right = 1in, top = 1in, bottom = 1in]{geometry}
\usepackage{bbm}  %% For making the blackboard "1" symbol.  Used as the identity function.
\usepackage{wrapfig}  %%  For embedding pictures in the text.
\usepackage{graphicx}  %%  For including pictures

%% ----------------------------------- 
%% ----- Course-Specific Macros ------
%% -----------------------------------

\theoremstyle{definition}
\newtheorem{definition}{Definition}
\newtheorem{exercise}{Exercise}
\newtheorem{example}{Example}
\newtheorem{remark}{Remark}
\newtheorem{corollary}{Corollary}


\theoremstyle{theorem}
\newtheorem{theorem}{Theorem}
\newtheorem{lemma}{Lemma}

%% ----- Math symbols ----------------
\newcommand{\R}{\mathbb{R}}
\newcommand{\N}{\mathbb{N}}
\newcommand{\Q}{\mathbb{Q}}
\newcommand{\one}{\mathbbm{1}}

%% ----------------------------------- 
%% -------- Titlepage Info -----------
%% -----------------------------------

\title{GRSC 7770:  A Field Guide to Teaching MATH 1113}
\speaker{William E. Olsen}
\contact{\href{mailto:wolsen@uga.edu}{wolsen@uga.edu}}
\place{University of Georgia}
\date{Fall, 2018} 
\online{The latest version is online \href{http://williamolsen.github.io}{here}.}

%% ----------------------------------- 
%% --------- Document Begins ---------
%% -----------------------------------

\begin{document}
\maketitle

{\pagestyle{plain}
\tableofcontents
\cleardoublepage}

\newpage

{\pagestyle{plain}
\listoflectures
\cleardoublepage}

\newpage


%% ----------------------------------- 
%% ----- Individual Lectures ---------
%% -----------------------------------
%% Lectures are included individually for fast compiling 
%% when live-TeX-ing. 




% !TEX root = sample-notes.tex
\lecture{1}{13 August 2018}
\section{Overview}

Here is a brief overview of the course.

\subsection{Pedagogy}
\label{sec:pedagogy}

Teaching pedagogy is hard for several reasons

\begin{enumerate}
\item It's likely that you don't know much about it.
\item There's a surprisingly large amount of literature, and it's hard to navigate.
\end{enumerate}

\subsection{Concrete outcomes}
\label{sec:concrete-outcomes}

There are several concrete outcomes that should be produced during this course.  They are:

\begin{enumerate}
\item Every student should have a website posted on the UGA directory.  With a thumbnail picture.
\item This website should have a CV.
\item Every student should have created their own syllabus, and be familiar with the rules of syllabi.
\item Every student should have a 10 minute video of them teaching.
\item Every student should have evalulated another professor on their teaching abilities.
\end{enumerate}

Other activities which I think are important

\begin{enumerate}
\item Have \emph{cheating} day.
\item Have \emph{syllabus} day.
\item Have other graduate students come in and have a panel.
\item Have website with teaching advice on it.  'What I wish I knew the first time I taught'\ldots
\item Much of what this course is collecting information and resources for students.
\item Have \emph{grading} day (probably two!)
\end{enumerate}

%%% Local Variables:
%%% mode: latex
%%% TeX-master: "../../trisection-notes"
%%% End:



\section{A Call to Action}
\label{sec:call-action}

Teaching STEM well is an imperative.  It's important.  Do it.

%%% Local Variables:
%%% mode: latex
%%% TeX-master: "../../trisection-notes"
%%% End:


\lecture{3}{17 August 2018}

\section{Teaching Models}
\label{sec:teaching-models}

We should teach models using \emph{active learning}. Meta!!
%%% Local Variables:
%%% mode: latex
%%% TeX-master: "../../trisection-notes"
%%% End:




\end{document}

%%% Local Variables:
%%% mode: latex
%%% TeX-master: t
%%% End:
%%% Local Variables:
%%% mode: latex
%%% TeX-master: t
%%% End:
