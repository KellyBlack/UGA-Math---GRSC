\documentclass{article}
\usepackage{termcal}

% Few useful commands (our classes always meet either on Monday and Wednesday 
% or on Tuesday and Thursday)

\newcommand{\MWClass}{%
\calday[Monday]{\classday} % Monday
\skipday % Tuesday (no class)
\calday[Wednesday]{\classday} % Wednesday
\skipday % Thursday (no class)
\skipday % Friday 
\skipday\skipday % weekend (no class)
}

\newcommand{\TRClass}{%
\skipday % Monday (no class)
\calday[Tuesday]{\classday} % Tuesday
\skipday % Wednesday (no class)
\calday[Thursday]{\classday} % Thursday
\skipday % Friday 
\skipday\skipday % weekend (no class)
}

\newcommand{\Holiday}[2]{%
\options{#1}{\noclassday}
\caltext{#1}{#2}
}


\begin{document}
\paragraph*{Tentative Schedule:}
\begin{center}
\begin{calendar}{1/11/2010}{16} % Semester starts on 1/11/2010 and last for 16
                    % weeks, including finals week
\setlength{\calboxdepth}{.3in}
\TRClass
% schedule
\caltexton{1}{1.1, 1.2 Review}
\caltextnext{1.3, 1.4 Review}
\caltextnext{2.1, 2.2 Linear Equations}
% ... and so on

% Holidays
\Holiday{1/18/2010}{Martin Luther King Day}
\Holiday{3/8/2010}{Spring Break}
% ... and so on

\options{4/26/2010}{\noclassday} % finals week
\options{4/27/2010}{\noclassday} % finals week
\options{4/28/2010}{\noclassday} % finals week
\options{4/29/2010}{\noclassday} % finals week
\options{4/30/2010}{\noclassday} % finals week
\caltext{4/27/2010}{\textbf{Final Exam}}
\end{calendar}
\end{center}
\end{document}

%%% Local Variables:
%%% mode: latex
%%% TeX-master: t
%%% End:
