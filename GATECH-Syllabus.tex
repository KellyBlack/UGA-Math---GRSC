\documentclass[10pt]{article}
%
\setlength{\parskip}{6 pt}
\setlength{\parindent}{0 pt}
%
\begin{document}
%
\begin{center}
{\bfseries \Large XX0000 Syllabus}\\
{\bfseries \large Course Name, Section X, Y Credits\\
Class Day(s), Time, Location (include lab/recitation locations)}
\end{center}

\section*{Instructor and TA Information}
\begin{tabular}[h] {l l l}
{\bfseries Name} & {\bfseries E-mail} & {\bfseries Office Hours}\\ \hline
{\bfseries Instructor:} & & \\ %Insert Instructor's name; Instructor's E-mail address; Instructor's office hours (time & place)
{\bfseries TA:} & & %Insert TA name; TA E-mail address; TA office hours (time & place)
\end{tabular}

\section{General Information}
\subsection*{Description}
%Your course description should provide a brief introduction to the scope, purpose,
%and relevance of the course. Note also that the course description in your syllabus
%can go beyond the description in the course catalogue, provided it is consistent
%with that description. Aim to give students a sense of what is interesting/useful 
%about the course, while avoiding the use of jargon and terms that students who
%haven’t yet taken the course might not understand.
%
\subsection*{Pre- and/or Co-Requisites}
%If applicable, list pre-requisites here. In some instances you may also want to
%describe the background knowledge/experience that is most likely to lead to success
%in your course (this is often relevant in a graduate level seminar and upper-level
%elective courses).
%
\subsection*{Course Goals and Learning Outcomes}
%Developing learning objectives is an important first step in course design, and they
%should be articulated on your syllabus as a bulleted list. Your learning objectives
%are meant to identify your main course goals for your students, in terms of the skills
%and knowledge they will develop in your class. They should be student-centered, action-
%oriented, and measurable, and they should reflect a big-picture view of the purpose of
%the course. One way to do this is to write them as a bulleted list of completions of
%this sentence starter: “Upon successful completion of this course, you should be able
%to…”. Aim for 3-5 learning objectives for a single course.

\section{Course Requirements and Grading}
%Use this chart to list a summary of the graded components in your class. Keeping
%the chart on the first page of your syllabus will help your students quickly answer
%their most pressing questions on the first day of class: What’s the workload like?
%How do I get an A in this course?
%Note also that you are expected to return a graded assignment or other meaningful
%performance feedback to your students prior to the deadline for withdrawing from
%classes – so that students can make informed decisions about withdrawal and their
%grade mode. In general it is good practice to give students multiple low stakes
%opportunities for performance assessment, prior to larger, high stakes events in
%your course.

%
\begin{tabular}[h]{lll}
{\bfseries Assignment} & {\bfseries Date} & {\bfseries Weight}\\ \hline
&&\\
&&\\
\end{tabular}
\subsection*{Extra Credit and Grade Dispute Policies and Procedures}
%If applicable, include a statement about opportunities for extra credit and grade
%dispute policies here. Views on extra credit opportunities vary among faculty.
%You might decide not to offer extra credit opportunities because you want your
%students to work hard in class and spend time working on actual assignments, or
%because you think extra credit lowers academic standards. However, extra credit
%can also be a good learning opportunity because it gives students an additional
%chance to learn the material (especially students who are struggling in the course).
%It also reduces student anxiety and builds their motivation and confidence.
%
\subsection*{Description of Graded Components}
%Your syllabus should include extra details and information for each component
%of your students’ final grade. The idea here is to give your students a sense of
%what kind and quantity of work will be expected of them. Some things can be lumped
%together (e.g. midterms & final exams), but there should be a descriptive blurb
%associated with every component of your course that counts toward a student’s
%final grade. In cases where attendance and/or participation will be graded, you
%should explain how you will be assessing their attendance and/or participation
%(visit http://ctl.gatech.edu/resources/syllabus/policies for examples).
%You can also include information about late-work policies, coursework resubmission,
%and information about how, where, and when assignments should be turned in.
%Finally, you should be sure to include information about how and where students
%are expected to turn in regular assignments.
%
\subsection*{Grading Scale}
Your final grade will be assigned as a letter grade according to the following scale:\\
A	90-100\%\\
B	80-89\%\\
C	70-79\%\\
D	60-69\%\\
F	0-59\%\\

%At Georgia Tech, grades are awarded on a scale of A-F with no +/- grades permitted.
%The grading scale inserted above is a standard option, but you are permitted to 
%adjust your approach based on the needs and design of your particular course.
%In your syllabus you should define your approach to assigning grades so that students
%can clearly see the ways in which their work and grades earned along the way will
%contribute to their final grade in the course. Further, you should try to avoid
%situations where (for example) very low grades through the semester are translated
%into As and Bs at the end of the semester: it is good practice to aim for alignment
%between student grades along the way, and final grades assigned at the end of the
%semester.
%According to policy, grades at Georgia Tech are interpreted as follows:
%	A	Excellent (4 quality points per credit hour)
%	B	Good (3 quality points per credit hour)
%	C	Satisfactory (2 quality points per credit hour)
%	D	Passing (1 quality point per credit hour)
%	F	Failure (0 quality points per credit hour)
%See http://registrar.gatech.edu/info/grading-system for more information about the grading system at Georgia Tech.

\section{Course Materials}
\subsection*{Course Text}
%List required course text books here, along with information on where to purchase/acquire them.
%
\subsection*{Additional Materials/Resources}
%If applicable, include items like lab supplies and other materials that are required
%for your class. Alternatively, consider including optional/support materials, like
%additional books/readings that interested and/or motivated students might want to read.
%
\subsection*{Course Website and Other Classroom Management Tools}
%Either mention your use of a Canvas site, or add a link to your course website.
%If you are using other classroom management tools (e.g. Turning Point Clickers), include
%information about those here.

\section{Course Expectations and Guidelines}
%In agreement with both best practices for teaching and learning and Georgia Tech policies
%and procedures, there are six types of policies that should be articulated in every Georgia
%Tech syllabus.  In addition to the content below, you can find more sample policies, more
%information about Georgia Tech specific rules and regulations, and more suggestions for what
%to consider when setting each policy, by visiting our Course Policies page
%(http://ctl.gatech.edu/resources/syllabus/policies).
%
%As you write this portion of your syllabus, use language that emphasizes your students’
%role in the process, and aim for a tone that communicates both authority and approachability.
%Each policy should make it clear what is and is not expected/permissible in this class, what
%the rationale or motivation behind the policy is, what students need to do in extenuating
%circumstances, and what the consequences will be if they fail to live up to the expectations
%laid out in the policy. Finally, your policy should represent something that you are
%comfortable implementing consistently throughout the course.

\subsection*{Academic Integrity}
Georgia Tech aims to cultivate a community based on trust, academic integrity, and honor. Students are expected to act according to the highest ethical standards.  For information on Georgia Tech's Academic Honor Code, please visit http://www.catalog.gatech.edu/policies/honor-code/ or http://www.catalog.gatech.edu/rules/18/.

Any student suspected of cheating or plagiarizing on a quiz, exam, or assignment will be reported to the Office of Student Integrity, who will investigate the incident and identify the appropriate penalty for violations.
%
\subsection*{Accommodations for Students with Disabilities}
If you are a student with learning needs that require special accommodation, contact the Office of Disability Services at (404)894-2563 or http://disabilityservices.gatech.edu/, as soon as possible, to make an appointment to discuss your special needs and to obtain an accommodations letter.  Please also e-mail me as soon as possible in order to set up a time to discuss your learning needs.
%
\subsection*{Attendance and/or Participation}
%Whether attendance and/or participation are required and/or graded in your class is up
%to you – and your position on this is an important course design consideration. However,
%there are several questions worth thinking about as you make that decision, and as you
%articulate your policy for your syllabus. In particular, if a student skips every class
%but achieves an A in the course, will you be satisfied that they took part in the full
%learning experience? In addition, how will student absences affect the learning experience
%of other students in your course, and what resources do you have at your disposal for
%tracking and/or grading attendance and/or participation? Please also see
%http://www.catalog.gatech.edu/rules/4/ for more information about institute expectations
%and restrictions around attendance, including information about excused absences.
%Instructors are also encouraged to consider the impact of events like the All-Majors
%Career Fair (http://www.careerdiscovery.gatech.edu/all-majors-career-fair), and off-campus
%interviews.
%
\subsection*{Collaboration and Group Work}
%The university’s Honor Code gives you the job of defining for your students the levels
%of collaboration that are permitted, as well what outside resources they are permitted
%to use (on assignments, exams, projects, etc.), and how they are supposed to report their
%use of those outside resources. Articulate your policy here.
%
\subsection*{Extensions, Late Assignments, and Re-Scheduled/Missed Exams}
%Students need to know what your policy is on things like late assignments and missed exams.
%You should be as clear as possible about your rules and the consequences for your students
%if they do not follow them.  You want to help students focus their efforts appropriately
%and also make it easy for you to be consistent throughout the course. Note also that at
%Georgia Tech, some exceptions are made for “approved Institute activities” (e.g. field
%trips and athletic events). See http://www.catalog.gatech.edu/rules/4/ for more information.
%Note also that instructors are encouraged to consider the impact of events like the All-Majors
%Career Fair (http://www.careerdiscovery.gatech.edu/all-majors-career-fair), and off-campus
%interviews, and to plan accordingly.
%
\subsection*{Student-Faculty Expectations Agreement}
At Georgia Tech we believe that it is important to strive for an atmosphere of mutual respect, acknowledgement, and responsibility between faculty members and the student body. See http://www.catalog.gatech.edu/rules/22/ for an articulation of some basic expectation that you can have of me and that I have of you. In the end, simple respect for knowledge, hard work, and cordial interactions will help build the environment we seek. Therefore, I encourage you to remain committed to the ideals of Georgia Tech while in this class.
%
\subsection*{Student Use of Mobile Devices in the Classroom}
%To set this policy, think about individual students, the overall dynamic you would like
%to see at work in your classroom, and your own tolerance of distractions in the classroom.
%See our Course Policies page for more information about factors to consider when it comes
%to setting your policy for the use of mobile devices in your classroom
%(http://ctl.gatech.edu/resources/syllabus/policies).
%
\subsection*{Additional Course Policies}
%There are a variety of additional policies you might include in your in your syllabus,
%depending on your specific context and approach to your course. For example, many instructors
%include at least one of the following policies explicitly on their syllabus: 
%   • accommodations for religious observances
%   • food and drink in the classroom
%   • freedom of expression and guidelines for discussion
%   • Institute-approved absences
%   • lab safety
%   • preparation for guest speakers
%   • re-grading and re-submission
%   • recording class activities
%Visit http://ctl.gatech.edu/resources/syllabus/policies for examples of additional course policies.

\section{Campus Resources for Students}
%Students might not be aware of all available campus resources. In this section you can 
%nclude specific resources that might help students succeed in you class (e.g. the library,
%The Communication Center, or The Center for Academic Success. Some faculty include resources
%that support students’ mental and emotional well-being (e.g. The Counseling Center, The
%Division of Student Life, or Women’s Resource Center). Including these additional resources
%on your syllabus communicates to students that you care about them and that you are committed
%to facilitating their academic progress. For a list of relevant campus resources available to
%Georgia Tech students, visit http://ctl.gatech.edu/sites/default/files/documents/campus_resources_students.pdf.

\section{Additional Syllabus Components}
%Depending on your specific context, as well as your own approach to your course and your
%teaching, you might decide to add other components to your syllabus. Research suggests
%that a more detailed syllabus is seen by students as a sign of teaching effectiveness,
%instructor approachability and flexibility, and as a motivating factor in class preparation.
%Additional information on your syllabus might include:
%   • a statement of your teaching philosophy;
%   • a statement about the importance of student mental health and well-being;
%   • rationale for your teaching techniques;
%   • grading rubrics;
%   • information about labs, recitations, etc.;
%   • advice on how to succeed in your course.


\pagebreak

\section*{Course Schedule}
%Include a clear course schedule for your students in this section. List dates of classes
%(including scheduled holidays and breaks), the content covered in each class and what students
%have to do in order to prepare, and due dates for assignments and exams. This section should
%help students stay organized during the semester. The section should also help establish a
%cognitive framework that helps students organize knowledge and skills they will acquire in
%the course. For a list of dates for the upcoming semester, organized by days of the week
%that your course is offered, visit the “List of GT Instructional Dates” section on our Syllabus
%page (http://ctl.gatech.edu/resources/syllabus).
\begin{tabular}[h]{lll}
{\bfseries Date}&{\bfseries Topic}&{\bfseries Readings, Notes, Due Dates, and more}\\ \hline
&&\\
\end{tabular}

%ALTERNATIVE FORMAT
%\beging{tabular}[h]{lllll}
%{\bfseries Date}&{\bfseries Prepare before class}& {\bfseries Topic during class}&{\bfseries Homework}&{\bfseires Assignments Due}\\ \hline
%&&&&\\
%\end{tabular}

%
\end{document}
%%% Local Variables:
%%% mode: latex
%%% TeX-master: t
%%% End:
