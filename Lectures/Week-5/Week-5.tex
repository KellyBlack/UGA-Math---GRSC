
\section{Week 5}
\label{sec:week-5}

\subsection{Problem solving in STEM}
\label{sec:problem-solving-stem}

\subsubsection{How do I decide which problems to cover in section or class?}
\label{sec:how-do-i}

In-class problem solving should reinforce the major concepts from the class and provide the opportunity for theoretical concepts to become more concrete. If students have a problem set for homework, then in-class problem solving should prepare students for the types of problems that they will see on their homework. You may wish to include some simpler problems both in the interest of time and to help students gain confidence, but it is ideal if the complexity of at least some of the in-class problems mirrors the level of difficulty of the homework. You may also want to ask your students ahead of time which skills or concepts they find confusing, and include some problems that are directly targeted to their concerns.

\subsubsection{You have given your class some problems to solve}
\label{sec:you-have-given}

Try to give your students a chance to grapple with the problems as much as possible.  Offering them the chance to do the problem themselves allows them to learn from their mistakes in the presence of your expertise as their teacher. (If time is limited, they may not be able to get all the way through multi-step problems, in which case it can help to prioritize giving them a chance to tackle the most challenging steps.)
When you do want to teach by solving the problem yourself at the board, talk through the logic of how you choose to apply certain approaches to solve certain problems.  This way you can externalize the type of thinking you hope your students internalize when they solve similar problems themselves.
Start by setting up the problem on the board (e.g you might write down key variables and equations; draw a figure illustrating the question).  Ask students to start solving the problem, either independently or in small groups.  As they are working on the problem, walk around to hear what they are saying and see what they are writing down. If several students seem stuck, it might be a good to collect the whole class again to clarify any confusion.  After students have made progress, bring the everyone back together and have students guide you as to what to write on the board.
It can help to first ask students to work on the problem by themselves for a minute, and then get into small groups to work on the problem collaboratively.
If you have ample board space, have students work in small groups at the board while solving the problem.  That way you can monitor their progress by standing back and watching what they put up on the board.
If you have several problems you would like to have the students practice, but not enough time for everyone to do all of them, you can assign different groups of students to work on different – but related - problems.

\subsection{When do you want students to work in groups to solve problems?}
\label{sec:when-do-you}

\begin{itemize}
\item Don’t ask students to work in groups for straightforward problems that most students could solve independently in a short amount of time.
  
\item Do have students work in groups for thought-provoking problems, where students will benefit from meaningful collaboration.
  
\item Even in cases where you plan to have students work in groups, it can be useful to give students some time to work on their own before collaborating with others.  This ensures that every student engages with the problem and is ready to contribute to a discussion.
\end{itemize}


%%% Local Variables:
%%% mode: latex
%%% TeX-master: "../../GRSC-notes"
%%% End:
