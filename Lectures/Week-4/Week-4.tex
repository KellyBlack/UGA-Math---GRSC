
\section{Week 4}
\label{sec:week-4}

\subsection{Grading the syllabus}
\label{sec:grading-syllabus}

Have students exchange syllabi and grade each other's work.  Of course, this is simply a completion grade as far as the GRSC-grade is concerned.  If a student fails to submit a syllabus, then their grade counts as an absence from the class.

\subsection{Discussion: Introduction to active learning}
\label{sec:discussion}

Active learning strategies come in many varieties, most of which can be grafted into existing courses without costly revisions. One of the simplest and most elegant exercises, called Think-pair-share, could easily be written into almost any lecture. In this exercise, students are given a minute to think about—and perhaps respond in writing—to a question on their own.  Students next exchange ideas with a partner.  Finally, some students share with the entire class. A think-pair-share engages every student, and also encourages more participation than simply asking for volunteers to respond to a question.

Other active learning exercises include:

\begin{itemize}
\item \textbf{Case studies:}  In a case study, students apply their knowledge to real life scenarios, requiring them to synthesize a variety of information and make recommendations.
  
\item \textbf{Collaborative note taking:}  The instructor pauses during class and asks students to take a few minutes to summarize in writing what they have just learned and/or consolidate their notes.  Students then exchange notes with a partner to compare; this can highlight key ideas that a student might have missed or misunderstood.
  
\item \textbf{Concept map:}  This activity helps students understand the relationship between concepts. Typically, students are provided with a list of terms.  They arrange the terms on paper and draw arrows between related concepts, labeling each arrow to explain the relationship.
  
\item \textbf{Group work:}  Whether solving problems or discussing a prompt, working in small groups can be an effective method of engaging students.  In some cases, all groups work on or discuss the same question; in other cases, the instructor might assign different topics to different groups.  The group’s task should be purposeful, and should be structured in such a way that there is an obvious advantage to working as a team rather than individually.  It is useful for groups to share their ideas with the rest of the class—whether by writing answers on the board, raising key points that were discussed, or sharing a poster they created.
  
\item \textbf{Jigsaw:}  Small groups of students each discuss a different, but related topic. Students are then shuffled such that new groups are comprised of one student from each of the original groups. In these new groups, each student is responsible for sharing key aspects of their original discussion. The second group must synthesize and use all of the ideas from the first set of discussions in order to complete a new or more advanced task.  A nice feature of a jigsaw is that every student in the original group must fully understand the key ideas so that they can teach their classmates in the second group. 

  
\item \textbf{Minute paper, or quick write:}  Students write a short answer in response to a prompt during class, requiring students to articulate their knowledge or apply it to a new situation.
  NB: A minute paper can also be used as a reflection at the end of class.  The instructor might ask students to write down the most important concept that they learned that day, as well as something they found confusing.  Targeted questions can also provide feedback to the instructor about students’ experience in the class.
  
\item \textbf{Statement correction, or intentional mistakes:}  The instructor provides statements, readings, proofs, or other material that contains errors.  The students are charged with finding and correcting the errors.  Concepts that students commonly misunderstand are well suited for this activity.
  
\item \textbf{Strip sequence, or sequence reconstruction:} The goal of this activity is for students to order a set of items, such as steps in a biological process or a series of historical events.  As one strategy, the instructor provides students with a list of items written on strips of paper for the students to sort.  Removable labels with printed items also work well for this activity.

  
\item \textbf{Polling:}  During class, the instructor asks a multiple-choice question.  Students can respond in a variety of ways.  Possibilities include applications such as PollEverywhere or Learning Catalytics.  In some courses, each student uses a handheld clicker, or personal response device, to record their answers through software such as TurningPoint or iClicker.  Alternatively, students can respond to a multiple-choice question by raising the appropriate number of fingers or by holding up a colored card, where colors correspond to the different answers. A particularly effective strategy is to ask each student to first respond to the poll independently, then discuss the question with a neighbor, and then re-vote.

\end{itemize}

%%% Local Variables:
%%% mode: latex
%%% TeX-master: "../../GRSC-notes"
%%% End:
