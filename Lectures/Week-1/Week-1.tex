\section{Week 1}
\label{sec:week-1}

\subsection{Introduction to GRSC 7770}
\label{sec:intr-grsc-7770}

First, introduce yourself. What is your background? What are your credentials? What do you find genuinely interesting about the course, and what are some of your other interests? The more students are able to connect with you, the easier your job will be. Similarly, the more you know about them, the better.  Here are some tips:

\begin{itemize}
\item Get to know your students. Ask your students to share with you why they are taking GRSC 7770 and what they hope to get out of it. Find out what their previous experience with the subject is. Learn their names. 
\item Encourage office hours by making one visit mandatory in the first month. Knowing more about your students will make it easier for you to teach them. 
\item Remember that you may be nervous as a teacher, but the students may also be nervous about a new class, new teacher, and potentially very new material. Breaking the tension right away can help to put everyone at ease.
\item Build rapport among the group.  Give your students opportunities to work with each other. Ask them to move around and work with different partners throughout a session.  This is especially important for GRSC 7770 because this class requires a lot of participation. 
\item Be sure to start with icebreakers to get things comfortable. It can be as simple as having students introduce themselves to a partner and then introduce their partner to the larger group.
\end{itemize}

\subsubsection{Other resources for building rapport with students}
\label{sec:other-reso-build}

Here a few outside resources for developing rapport with students and building a positive classroom climate:
\begin{itemize}
\item \href{run:/Lectures/Week-1/BuildingRapport.pdf}{Simple strategies to develop rapport with students and build a positive classroom climate}
\end{itemize}



\subsection{Covering the syllabus}
\label{sec:covering-syllabus}

You need to set expectations and cover the syllabus on the first day.  You may have a few students from other departments, and its important for them to know that this course is geared primarily towards the mathematics department.  Important features of the course are:

\begin{itemize}
\item Attendance is mandatory.  If a student misses too many classes (you decided), then they automatically fail the class.  Unfortunately, you'll have to keep up constant pressure about this throughout the semester because otherwise the class attendance will plummit.
\item The students are expected to complete all assignments.  Again, you'll have to maintain pressure throughout the semester.
\item Late assignments will not be accepted.
\item The textbook \href{https://www.maa.org/sites/default/files/InstructPracGuide_web.pdf}{MAA Instructional Practices Guide} is free online.
\item UGA, and the mathematics department in particular, has lots of support for teachers; such additional resources like the CTL, Kelly Black, and MALT.
\end{itemize}

\subsection{Discussion: A Declaration of Values}
\label{sec:disc-decl-valu-1}

The first \emph{daily discussion} will be lead by you and will focus on the reading assigned (hopefully).  This is a great opportunity for you to set a benchmark.  In particular, you should do the following:
\begin{itemize}
\item Have a 5-7 minutes presentation on \emph{A declaration of values}.
\item Include an active learning component.
\item Include a reflection component.
\end{itemize}

This will show the GRSC students a full cycle in the teaching process.


%%% Local Variables:
%%% mode: latex
%%% TeX-master: "../../GRSC-notes"
%%% End:
