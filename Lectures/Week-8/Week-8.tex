
\section{Week 8}
\label{sec:week-8}

\subsection{Discussion: Making the grade}
\label{sec:making-grade}

This week's reading is pages 59--63.  

\subsection{Activity}
\label{sec:activity}

Today's activity is to have the GRSC students grade a quiz.  

\subsection{Some general principles of responding to student work}
\label{sec:some-gener-princ}

Here are the most important.

\begin{itemize}
\item \textbf{Know Your Goals and Name Them.} Grading allows students to know where they stand in relation to learning goals, whether they’re the goals of a given assignment, a sequence of assignments, a semester-long course, or a broader course of study. Therefore, the first step in effective grading happens before students start writing a paper or sit down with a problem set: you need to decide what your learning goals are, name them, and identify what criteria will allow you to measure student progress toward those goals.
  
\item \textbf{Be Transparent with Students.} As necessary as learning goals and concrete criteria are, the goal isn’t just to have them—they need to be shared with your students. Getting students on the same page with you about why they are doing an assignment and how they’re being assessed is a crucial part of making graded feedback purposeful, and the more transparent the goals and criteria are, the better. Starting with the course description and the syllabus, and extending through the prompts for assignment sequences and on through capstone projects or final exam, transparency allows the grading process to be more of a dialogue than a judgment from on high.
  
\item \textbf{Strive for Consistency.} At some point—after you’ve framed an assignment for yourself and your students, and after they’ve uploaded their assignment or put down their pencils—the time for grading will arrive. In an ideal world free of miscommunication or disciplinary foibles, the next step would simply be to block off chunks of time and apply the rubric you’d shared with your students beforehand. The world being what it is, however, challenges often arise. For instructors grading essay assignments, a common challenge is helping students see that your qualitative assessment is consistent, i.e., that it isn’t just a matter of your taste or preference or mood. Anyone who’s taught in the humanities or social science is likely to have stories about students who feel as though their grade on an essay assignment was just a reflection of the instructor’s subjective or impressionistic response, or just a measure of how closely the essay’s thesis came to the instructor’s own position on some matter of academic dispute. And to be fair, in the absence of clear learning goals and a rubric that’s been consistently applied, it’s hard to dispel that skepticism. The question here is what it means to apply a rubric consistently, and the short answer is this:

  \begin{itemize}
  \item Read through the lens of the criteria you’ve established in your prompt and rubric (thesis, identifying positions within a debate, use of secondary sources)
    
  \item Show your priorities by focusing on those criteria in your marginal feedback (don’t get bogged down with comments on style or structure if those aren’t tied to your learning goals)
    
  \item Organize your feedback letter in terms of your rubric’s criteria, so that the letter itself becomes an evidence-based argument that supports your claim about how successfully the student's written product  did or didn’t demonstrate mastery of the skills described in the assignment’s learning goals
  \end{itemize}
  
\item \textbf{Offer Context.} For instructors grading assignments that typically receive number grades and have answers on a clearer right/wrong spectrum (vocab quizzes, math problem sets, short-answer ID tests), the objectivity or consistency of the instructor is less likely to come into question. In these cases, by contrast, the challenge for instructors is moving past the idea that having an objective rubric alone makes their grading process is meaningful or fair. A grade on a page can’t speak for itself, and for it to be meaningful it needs to correspond to students’ learning contexts, e.g., how does the grade reflect students’ engagement with class time and course materials, or their preparation for the test, or their use of office hours, etc. And for a grade to be genuinely fair, it needs to correspond to pathways to future success. One effective tool for lending context to quantitative feedback is the exam wrapper, a guided set of questions students complete after receiving their graded work that asks them to identify their own areas of understanding and confusion, to reflect on how they prepared for the exam, and, often, to provide instructor feedback on how effectively the exam indeed tested mastery of the goals it set out to measure.

\end{itemize}

%%% Local Variables:
%%% mode: latex
%%% TeX-master: "../../GRSC-notes"
%%% End:
