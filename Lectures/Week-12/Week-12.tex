\section{Week 12}
\label{sec:week-12}

\subsection{Learner-centered design}
\label{sec:learn-cent-design}

Good course design always begins with clarity about your basic commitments and responsibilities as an instructor. In our experience, all courses, regardless of their discipline, size, or position within the curriculum, share the following five features:

\begin{itemize}
\item a \textbf{defined topic}, e.g. "Organic Chemistry," or "Gender in the Modern Novel."
  
\item some body of \textbf{curated content.} (This would include items like the readings, websites, films, or objects with which students are expected to interact, whether in class or on their own time.)

  
\item some number of \textbf{demonstrations} of how an expert (i.e. the instructor) can analyze/interpret/process that curated content in order to make claims about the course’s topic. (These demonstrations, which might comprise anything from the interpretation of experimental data to the close reading of a literary text, typically occur in lectures and class discussions.)

  
\item some number of opportunities to generate \textbf{evidence} of students' progress towards mastery. (These opportunities usually take the form of assignments, e.g., exams or papers, though they might also be folded into class discussions or—in a lecture setting, for example—real-time surveys.)
  
\item last, but certainly not least, \textbf{students}.
\end{itemize}
  
Whether or not they think about it in precisely these terms, every instructor who creates a course syllabus engages in the process of making dozens, if not hundreds, of decisions about the nature, purpose, relationship, and importance of each of these five features. In so doing, instructors might be guided by any one of a large number of possible criteria or incentives, ranging from their own experiences of being a student, to the expectations of their disciplines, to the fiscal constraints of their institutions. Sometimes, especially when instructors have become so expert in their fields as to be unable to "unthink" how or why they know some lecture topic to be essential or some exercise to be effective, these choices may not feel like choices.


When taken together, the hundreds of decisions which instructors make—about the kinds of material that will feature in their courses, the kinds of class activities they will deploy, and the kinds of evidence they will collect about their students' learning—go a long way towards defining how their courses may come across to a student audience. Many courses might be called instructor-centered, insofar as their instructors have designed with themselves—that is, with their identities, preferences, and peers—in mind. Such courses often put a premium on (1) offering comprehensive coverage of their topic, and (2) presenting it in a way which other scholars knowledgeable about the field would consider elegant, insightful, provocative, or clever. This approach is often especially common at research-intensive universities, where faculty bring frameworks adapted from their own research to bear on the question of how to relate or sequence different concepts in innovative and insightful ways. While these instructor-centered courses may provide students wonderful opportunities to learn, they may also present limitations when it comes to accounting for that student learning. Because their syllabi may be doing more justice to their topics (not to mention the vicissitudes of the academic calendar) than to their students, the assignments which students are asked to complete may feel like an afterthought—like something added only after the content has been sequenced.


At the Bok Center, we urge instructors to consider making a conscious decision to pursue a design process that is explicitly, intentionally learner-centered. We do so not because we want you to abandon your strengths as a faculty member at a research-intensive institution, but rather because we think that you will meet with greater success in the classroom if you foreground a consideration for students’ prior knowledge, their experience, and their learning when making decisions about your course’s most salient features. In other words, without changing how you think about the features of your course, we hope you will think explicitly about how students relate to them. To revisit our earlier list, in learner-centered design, all courses share:

\begin{itemize}
\item a \textbf{defined topic}. (Why do I want to teach this? What will students be able to do or say as a result of taking this course? In five weeks—or five years—from now, what do I want my students to be able to do or want to do as a result of my course?)

  
\item opportunities to generate \textbf{evidence} of students' progress towards mastery. (How will you know that your students can do the things you have set out to teach? When, and how, will you provide them with practice and feedback in their progress towards mastery?)
  
\item some body of \textbf{curated content}. (Why this content? Why not something else? What makes it valuable? Is it a case study? How does it relate to the other content?)
  
\item some number of \textbf{demonstrations} of how an expert can analyze/interpret/process that curated content in order to make claims about your topic. (What methods, theories, or pedagogical techniques allow students to address the topic in a richer or innovative way?)
\end{itemize}


Perhaps the most significant difference between this list and the version with which we began is how it is ordered. Evidence of student learning, previously relegated to the end of the list—and, more often than not, to the very last step of the design process, as instructors sprinkle in a handful of quizzes, papers, and midterms—is now, in our student-centered list, the very first consideration after the definition of the course’s topical focus. To instructors who are unfamiliar with this approach to course design, starting in this fashion—starting, in other words, with the kinds of assignments you will need in order to measure what your students have learned—may feel "backwards." (Indeed, in the teaching and learning world, this approach to course design which begins with goals and evidence before moving on to content is called "backward design.") How, you might ask, can I possibly know what kinds of evidence of student learning I can collect before I even know which poems or textbook chapters my students will have read? With a little practice, however, we think it will become apparent why we start with the question of evidence. After all, can you really say that you have taught something if you can’t show that your students have learned it?

%%% Local Variables:
%%% mode: latex
%%% TeX-master: "../../GRSC-notes"
%%% End:
