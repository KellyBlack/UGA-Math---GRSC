\section{Week 6}
\label{sec:week-6}

\subsection{What should you do while your students are working on problems?}
\label{sec:what-should-you}

\begin{itemize}
\item Walk around and talk to students. Observing their work gives you a sense of what people understand and what they are struggling with. Answer students’ questions, and ask them questions that lead in a productive direction if they are stuck.
  
\item If you discover that many people have the same question—or that someone has a misunderstanding that others might have—you might stop everyone and discuss a key idea with the entire class.
\end{itemize}

\subsection{After students work on a problem during class, what are strategies to have them share their answers and their thinking?}
\label{sec:after-students-work}

\begin{itemize}
\item Ask for volunteers to share answers. Depending on the nature of the problem, student might provide answers verbally or by writing on the board. As a variant, for questions where a variety of answers are relevant, ask for at least three volunteers before anyone shares their ideas.
  
\item Use online polling software for students to respond to a multiple-choice question anonymously.
  If students are working in groups, assign reporters ahead of time. For example, the person with the next birthday could be responsible for sharing their group’s work with the class.
  
\item Cold call. To reduce student anxiety about cold calling, it can help to identify students who seem to have the correct answer as you were walking around the class and checking in on their progress solving the assigned problem. You may even want to warn the student ahead of time: "This is a great answer! Do you mind if I call on you when we come back together as a class?"
  
\item Have students write an answer on a notecard that they turn in to you.  If your goal is to understand whether students in general solved a problem correctly, the notecards could be submitted anonymously; if you wish to assess individual students’ work, you would want to ask students to put their names on their notecard.
  
\item If you had assigned different groups to work on different problems, you can:
  \begin{itemize}
  \item Use a jigsaw strategy, where you rearrange groups such that each new group is comprised of people who came from different initial groups and had solved different problems.  Students now are responsible for teaching the other students in their new group how to solve their problem.
  \item Have a representative from each group explain their problem to the class.
  \item Have a representative from each group draw or write the answer on the board.
  \end{itemize}
\end{itemize}

\subsection{What happens if a student gives a wrong answer?}
\label{sec:what-happens-if}

\begin{itemize}
\item Ask for their reasoning so that you can understand where they went wrong.
\item Ask if anyone else has other ideas. You can also ask this sometimes when an answer is right.
\item Cultivate an environment where it’s okay to be wrong. Emphasize that you are all learning together, and that you learn through making mistakes.
\item Do make sure that you clarify what the correct answer is before moving on.
\item Once the correct answer is given, go through some answer-checking techniques that can distinguish between correct and incorrect answers. This can help prepare students to verify their future work.
\end{itemize}

\subsection{A few final notes}
\label{sec:few-final-notes}

\begin{itemize}
\item Make sure that you have worked all of the problems and also thought about alternative approaches to solving them.
\item Board work matters. You should have a plan beforehand of what you will write on the board, where, when, what needs to be added, and what can be erased when. If students are going to write their answers on the board, you need to also have a plan for making sure that everyone gets to the correct answer. Students will copy what is on the board and use it as their notes for later study, so correct and logical information must be written there.
\end{itemize}


%%% Local Variables:
%%% mode: latex
%%% TeX-master: "../../GRSC-notes"
%%% End:
