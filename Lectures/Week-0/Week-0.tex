
\section{Introduction}
\label{sec:introduction}

Congratulations on being this year's GRSC 7770 instructor in the mathematics department.  Because this course has no strictly set curriculum, we've written this guide to help organize your thoughts and share with you what's been done in the past.  Of course, you should feel free improve on this guide in any way you see fit.

\subsection{Structure and philosophy of GRSC 7770}
\label{sec:structure-course}

The GRSC 7770 seminar plays two important roles in the UGA mathematics department.  First and foremost, GRSC 7770 is a field guide to teaching MATH 1113 for the first time.  The emphasis is on students experiencing and experimenting with ``real life'' teaching techniques that they'll use next year while teaching MATH 1113.  Indeed, many of the activities scheduled on the syllabus are simply adaptations of what I do while teaching MATH 1113 or 2250 myself.  The structure of the course reflects the field guide philosophy via the ``study the micro to teach the macro'' approach (and is also reflected in our textbook \href{https://www.maa.org/sites/default/files/InstructPracGuide_web.pdf}{\emph{MAA Instructional Practices Guide}}):

\begin{enumerate}
\item \textbf{Classroom practices:}  This section provides concrete teaching practices, both inside and outside the classroom, that foster student engagement as well as a discussion on inclusivity and community in the classroom.
\item \textbf{Assessment practices:} This section centers on the interplay between formative and summative assessment to examine the teaching and learning of mathematics with a strong focus on learning outcomes.
\item \label{itm:DP} \textbf{Design practices:} This section provides a brief introduction to instructional design that help achieve desired learning outcomes, based on theories of design, along with potential challenges and opportunities associated with instructional design.
\end{enumerate}

The GRSC 7770 seminar plays a second (but equally important!) role as well; it is an extended welcome from the mathematics department to first-year graduate students.  We're happy that these students have come to join us, and they should know it.



\subsection{The nitty gritty of a day in GRSC 7770}
\label{sec:nitty-gritty-day}

This section focuses on what a ``typical day'' might look like in the GRSC 7770 seminar.  Looking at the syllabus, you'll see that each day comes in one of two flavors:
\begin{enumerate}
\item \label{itm:first} The first flavor is centered around student activities and participation (e.g. Weeks 1 and 2 on the syllabus).
\item \label{itm:second} The second flavor includes a guest speaker to come in and spice things up (e.g. Week 3).
\end{enumerate}

Usually, in a class of the first flavor \eqref{itm:first}, the class begins with something called a \emph{daily discussion}.  These are described in more detail in subsection \ref{sec:daily-discussions} below.  

\subsubsection{Daily discussions}
\label{sec:daily-discussions}

Looking at the syllabus, you'll see that most days have a ``daily discussion'' component.  A description of this phenomenon follows:

\begin{itemize}
\item A priori, the class should be partitioned into groups of size approx. 3.  One of the groups of students will lead the \emph{daily discussion} each week.
\item The group should spend 5-7 minutes at the board giving a group a presentation; they should provide an outline and summary of that week's reading.  
\item After the presentation, the group should lead an active learning activity together with the whole class.  This should last another 5-7 minutes.  
\end{itemize}

\subsubsection{Guest speakers}
\label{sec:guest-speakers}

Sometimes its best to let the professionals do the talking.  You should expect to invite 3-4 guest speakers to your GRSC 7770 seminar.  I recommend the following speakers:
\begin{itemize}
\item Philip Griffeth--Academic honesty
\item Dr. Michelle Cook--Teaching diverse students
\item Judy Milton--Teaching portfolios
\end{itemize}



\subsection{Preparing for your first day}
\label{sec:preparing-your-first}

The GRSC 7770 seminar requires a lot more preparation than other courses.  I recommend you do the following before the first day of class:

\begin{enumerate}
\item Check with the UGA \href{https://github.com/WilliamOlsen/GRSC-2018/blob/master/Lectures/Week-0/GRSC7770-Guidelines-2018.pdf}{GRSC 7770 course guidelines} to ensure that your course satisfies the requirements.
\item Rent the camera equipment from the CTL.
\item Schedule guest speaker dates throughout the semester.  This involves emailing them and confirming dates/times.
\item Email the GRSC class with the syllabus and a welcome email.  Also, this email should contain their first reading assignment: \emph{A Declaration of Values}. 
\item Always check out your room before your first day of class! 
\end{enumerate}


%%% Local Variables:
%%% mode: latex
%%% TeX-master: "../../GRSC-notes"
%%% End:
