\section{Week 2}
\label{sec:week-2}

\subsection{How to write a syllabus}
\label{sec:how-write-syllabus}

Today's activity is focused on the zen of writing syllabi.  This activity is a necessary evil because they will soon be responsible for writing their own syllabus for MATH 1113.  

\subsection{Key functions and components of a syllabus}
\label{sec:key-funct-comp}

A syllabus has several functions. The first function is to invite students to your course—to inform them of the objectives of the course and to provide a sense of what the course will be like. The second function is to provide a kind of contract between instructors and students—to document expectations for assignments and grade allocations. The third function is to provide a guiding reference—a resource to which students and instructional staff can refer for logistical information such as the schedule for the course and office hours, as well as rationale for the pedagogy and course content.

Generally, a syllabus should include the following information:

\begin{itemize}
\item \textbf{Learning Objectives.} What students will gain or take away from your course. Why these objectives are the most important skills/knowledge for the course (helpful if objectives are included for each topic/session).
  
\item \textbf{Goal/Rationale.} How the course relates to primary concepts and principles of the discipline (where it fits into the overall intellectual area). Type of knowledge and abilities that will be emphasized. How and why the course is organized in a particular sequence.

\item \textbf{Basic Information.} Course name and number, meeting time and place, instructor name, contact information, office hours, instructional support staff information.

\item \textbf{Course Content.} Schedule, outline, meeting dates and holidays, major topics and sub-topics preferably with rationale for inclusion.
  
\item \textbf{Student Responsibilities.} Particulars and rationale for homework, projects, quizzes, exams, reading requirements, participation, due dates, etc. Policies on lateness, missed work, extra credit, etc.

\item \textbf{Grading Method.} Clear, explicit statement of assessment process and measurements.
  Materials and Access. Required texts and readings, course packs. How to get materials including relevant instructional technologies. Additional resources such as study groups, etc.
  
\item \textbf{Teaching Philosophy.} Pedagogical approach including rationale for why students will benefit from it.
\end{itemize}

\subsection{Information from the CTL}
\label{sec:information-from-ctl}

Happily, the CTL provides us with some additional information.  For example:
\begin{itemize}
\item \href{https://github.com/WilliamOlsen/GRSC-2018/blob/master/Lectures/Week-2/UGA-Syllabus-Template-Instructors-of-GRSC7770.pdf}{A UGA approved sample syllabus}
\end{itemize}


%%% Local Variables:
%%% mode: latex
%%% TeX-master: "../../GRSC-notes"
%%% End:
