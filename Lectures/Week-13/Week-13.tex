\section{Week 13}
\label{sec:week-13}

\subsection{Inclusive course design}
\label{sec:incl-course-design}

When designing a course, each move matters. From your selection of course materials, to your teaching methods, to the ways you ask students to demonstrate their learning, your course may privilege some students while disadvantaging others. There are moves you can make during the course design phase, though, that can help you create a more equitable and inclusive learning experience.

\subsection{Plan to access early and often}
\label{sec:plan-access-early}

Assessment doesn’t always have to be in the form of high-stakes midterms, exams, or papers. There are a number of low-stakes techniques that are easy to implement and that can provide quick, useful information. For example, you might collect student information using index cards or polling software. At the start of a class or a new module of material, you can ask students to write answers to brief background knowledge questions to get a sense of where they’re starting from. Once or twice throughout the semester, ask students to write down and pass in their “muddiest point,” that is, the most confusing part of the class they’ve experienced thus far; ensure that their feedback remains anonymous. Providing ongoing opportunities for assessment can both inform you about how student learning is going and where it might be necessary to make adjustments, and it can allow students a chance to reflect on how well they’re learning and where they might adjust their approach to your course as needed. See more on ongoing assessments.

\subsection{Plan to vary teaching strategies}
\label{sec:plan-vary-teaching}

Planning to employ various teaching strategies can go a long way toward fostering inclusivity by validating different learning tendencies and understandings of what counts as knowledge. Some teaching strategies will work wonderfully for some, but not others. Consider using a combination of board work, slides, videos, comics, podcasts—a variety of media and activities. Rather than simply lecturing or running through problem sets, mix things up by getting students to come up with answers in small groups, or invite students to the board to show their process. For discussions, try having students talk in pairs or smaller groups before opening up to the whole group. There are many ways to mix things up.

\subsection{Allow students to demonstrate their learning in various ways, when possible}
\label{sec:allow-stud-demonstr}

Some students excel at discussing and articulating arguments in class; others may share deep insights during office hours or via online discussion platforms. Encourage students to develop in all areas and forms of expression, but also communicate your recognition that not all students demonstrate their grasp of the material in the same ways equally. If a student or group of students is having difficulty learning or communicating through a particular medium, then it can be helpful to dedicate class time to walking everyone through the purpose and process of working with that medium. For example, if students are to research and write an academic paper, then you may spend some class time clearly articulating the purpose of not just this particular paper, but academic paper writing more broadly. If you are inviting students to use a less conventional medium to complete an assignment—creating a video or podcast, for instance—then it’s important to articulate how these media provide unique opportunities for demonstrating learning, and why you believe working with them is a good fit for the learning aims of the assignment. This explicit communication about how learning can be demonstrated in various ways, through various media, helps convey to students your support and openness to the differences they may experience in demonstrating what they’re learning.


%%% Local Variables:
%%% mode: latex
%%% TeX-master: "../../GRSC-notes"
%%% End:
