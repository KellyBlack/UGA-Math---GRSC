\section{Week 14}
\label{sec:week-14}

\subsection{Inclusive moves}
\label{sec:inclusive-moves}

Students should have equitable opportunities for learning, regardless of their race, ethnicity, sexual orientation, gender, religion, linguistic or socioeconomic background, ability, and more. What concrete moves can we make to foster an optimal environment for learning, which encourages engagement, authenticity, and respect?

While course design is the first step to teaching equitably, cultivating a classroom climate that fosters learning is an ongoing process. It starts on the first day of class and entails creating, communicating, and managing the classroom norms and conventions in a way that is conducive to learning. The following strategies can help instructors achieve an inclusive classroom climate.

\subsection{Set the tone on your first day}
\label{sec:set-tone-your}

Set the tone for inclusivity by providing opportunities for students to introduce themselves, learn about their classmates, and learn about you. You’re all in the same classroom: what brought you to this particular classroom, at this particular time, to study this particular topic? What do you hope to learn or get out of the class? (Note: you’ll have to be open to receiving a variety of answers, including ones that indicate the class is required and they’re simply hoping to get some credits from it!) Clearly introduce students to the course’s learning goals and any methods they should expect to engage in regularly throughout the semester. If your class primarily involves discussion, for instance, then be explicit about discussion norms and conventions from day one. You may even come up with ground rules for group interaction together as a class; these ground rules can live on an online version of the syllabus or the course’s website. The first day is also a great time to articulate how you understand your course to be relevant to your students’ lives.

\subsection{Get to know your students}
\label{sec:get-know-your}

Beyond learning names, ask your students why they’re taking your course and what they hope to get out of it. Beyond asking these questions, find out about your students’ previous experience with your discipline and subject. You may also want to learn more about their extracurricular passions and commitments. You can collect all of this information on index cards on the first day, or you could invite students to sign-up for a brief one-on-one or small group meeting with you at the start of the semester. In any case, encourage your students to take advantage of your office hours throughout the semester. Requiring them to visit you at least once in the first month of the course—even briefly—can be a helpful signal of your availability and willingness to engage.

\subsection{Build rapport among the group}
\label{sec:build-rapport-among}

Give your students regular opportunities to interact with each other and you. Start classes with an icebreaker or opening ritual or routine to help the group feel comfortable and aligned. You may ask students to take turns leading the opening exercise so that ownership over this part of the class is shared. Throughout the semester, invite students to move around and work with different partners. This movement not only prevents classroom social dynamics from getting stale, it also allows students an opportunity to get to know each other throughout the class’ duration. This variation is especially important for discussion-based classes, or sections that require a lot of pair work.




%%% Local Variables:
%%% mode: latex
%%% TeX-master: "../../GRSC-notes"
%%% End:
