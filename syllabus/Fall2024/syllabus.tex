\documentclass[11pt, a4paper]{article}
%\usepackage{geometry}
\usepackage[inner=1.5cm,outer=1.5cm,top=2.5cm,bottom=2.5cm]{geometry}
\pagestyle{empty}
\usepackage{graphicx, multicol}
\usepackage{fancyhdr, lastpage, bbding, pmboxdraw}
\usepackage[usenames,dvipsnames]{color}
\definecolor{darkblue}{rgb}{0,0,.6}
\definecolor{darkred}{rgb}{.7,0,0}
\definecolor{darkgreen}{rgb}{0,.6,0}
\definecolor{red}{rgb}{.98,0,0}
\usepackage[colorlinks,pagebackref,pdfusetitle,urlcolor=darkblue,citecolor=darkblue,linkcolor=darkred,bookmarksnumbered,plainpages=false]{hyperref}
\renewcommand{\thefootnote}{\fnsymbol{footnote}}

\usepackage{longtable}
\usepackage{multirow}
\usepackage{fourier} 
\usepackage{array}
\usepackage{makecell}

\renewcommand\theadalign{bc}
\renewcommand\theadfont{\bfseries}
\renewcommand\theadgape{\Gape[4pt]}
\renewcommand\cellgape{\Gape[4pt]}


\pagestyle{fancyplain}
\fancyhf{}
\lhead{ \fancyplain{}{GRSC 7770} }
%\chead{ \fancyplain{}{} }
\rhead{ \fancyplain{}{\today} }
%\rfoot{\fancyplain{}{page \thepage\ of \pageref{LastPage}}}
\fancyfoot[RO, LE] {page \thepage\ of \pageref{LastPage} }
\thispagestyle{plain}

%%%%%%%%%%%% LISTING %%%
\usepackage{listings}
\usepackage{caption}
\DeclareCaptionFont{white}{\color{white}}
\DeclareCaptionFormat{listing}{\colorbox{gray}{\parbox{\textwidth}{#1#2#3}}}
\captionsetup[lstlisting]{format=listing,labelfont=white,textfont=white}
\usepackage{verbatim} % used to display code
\usepackage{fancyvrb}
\usepackage{acronym}
\usepackage{amsthm}
\VerbatimFootnotes % Required, otherwise verbatim does not work in footnotes!



\definecolor{OliveGreen}{cmyk}{0.64,0,0.95,0.40}
\definecolor{CadetBlue}{cmyk}{0.62,0.57,0.23,0}
\definecolor{lightlightgray}{gray}{0.93}



\lstset{
%language=bash,                          % Code langugage
basicstyle=\ttfamily,                   % Code font, Examples: \footnotesize, \ttfamily
keywordstyle=\color{OliveGreen},        % Keywords font ('*' = uppercase)
commentstyle=\color{gray},              % Comments font
numbers=left,                           % Line nums position
numberstyle=\tiny,                      % Line-numbers fonts
stepnumber=1,                           % Step between two line-numbers
numbersep=5pt,                          % How far are line-numbers from code
backgroundcolor=\color{lightlightgray}, % Choose background color
frame=none,                             % A frame around the code
tabsize=2,                              % Default tab size
captionpos=t,                           % Caption-position = bottom
breaklines=true,                        % Automatic line breaking?
breakatwhitespace=false,                % Automatic breaks only at whitespace?
showspaces=false,                       % Dont make spaces visible
showtabs=false,                         % Dont make tabls visible
columns=flexible,                       % Column format
morekeywords={__global__, __device__},  % CUDA specific keywords
}

%%%%%%%%%%%%%%%%%%%%%%%%%%%%%%%%%%%%
\begin{document}
\begin{center}
{\Large \textsc{GRSC 7770 Graduate Teaching Seminar}}
\end{center}
\begin{center}
Fall 2024
\end{center}
% \date{August 2020}


\begin{center}
\rule{6in}{0.4pt}
\begin{minipage}[t]{.75\textwidth}
\begin{tabular}{llcccll}
\textbf{Instructor:} & Michaela Coleman & & &  & \textbf{Time:} & TR 2:20 -- 3:35pm \\
\textbf{Email:} &  \href{Michaela Coleman}{Michaela.Coleman@uga.edu} & & & & \textbf{Place:} & Boyd 204
\end{tabular}
\end{minipage}
\rule{6in}{0.4pt}
\end{center}
\vspace{.5cm}
\setlength{\unitlength}{1in}
\renewcommand{\arraystretch}{2}

\textit{The course syllabus is a general plan for the course; deviations
announced to the class by the instructor may be necessary.}

\begin{description}
  
\item[Course Webpage] eLC


\item[Office Hours] After class, or by appointment.

\item[Main References] %\footnotemark
  This is a restricted list of useful books that will be touched on
  during the course. You need to consult them occasionally.
  \begin{itemize}
  \item The {\textit{MAA Instructional Practices Guide}}.
  \item Steven G. Krantz, {\textit{How to Teach Mathematics}}, American Mathematical Society, 1991.
  \end{itemize} 

% \footnotetext{Downloadable ebook versions are available on AeLP.}

\item[Course Description] The primary goal of this seminar is to
  prepare graduate students for their roles as teaching assistants
  (and graduate student teachers) in the Department of Mathematics. A
  portion of the seminar will be devoted to department and university
  policies and procedures related to teaching. We will also discuss
  the roles that teaching assistants play in the department and how to
  balance that role with other responsibilities. Most importantly, we
  study and discuss effective teaching practices for undergraduates
  and why these practices are important in both academic and
  professional positions.


\item[Specific learning goals] We will use a variety of formats (e.g.,
  small-group-discussion, presentations) to explore the ins and outs
  of teaching mathematics as a graduate student.  By the end of this
  course you will be introduced to:
  \begin{multicols}{2}
    \begin{itemize}
    \item how to set the stage for your first class.
    \item motivation in the college classroom.
    \item assessments and rubrics.
    \item teaching culturally diverse students.
    \item dealing with student problems and problem students.
    \item inquiry-based learning.
    \item teaching controversial issues.
    \item teaching opportunities at UGA.
    \end{itemize}
  \end{multicols}

\item[Grading Policy] Course grades will be assigned as either $S/U$
  (Satisfactory/Unsatisfactory). To receive an $S$, students must:
  \begin{itemize}
  \item Complete all required readings and assignments.
  \item Participate in class and provide peer feedback.
  \item Have no more than two unexcused absences.
  \end{itemize}


\item[Communication] To comply with the Family Educational Rights and
  Privacy Act (FERPA), all communication that refers to individual
  students must be through a secure medium (UGAMail or eLC) or in
  person. Instructors are not allowed to respond to messages that
  refer to individual students or student progress in the course
  through non-UGA accounts, phone calls, or other types of electronic
  media.


\item[Academic Honesty] It is each student’s responsibility to be
  familiar with University policy on academic honesty (read \textit{A
    Culture of Honesty: Policies and Procedures on Academic
    Honesty}). See
  \href{http://www.uga.edu/honesty/ahpd/culture_honesty.htm}{\url{http://www.uga.edu/honesty/ahpd/culture_honesty.htm}}. Any
  evidence of academic dishonesty will be turned over to the Office of
  the Vice President for Academic Affairs, for consideration and
  possible action.  All students have the right to appeal any decision
  following the appeals process outlined in \textit{A Culture of
    Honesty}.


\item[Disabilities] I am happy to accommodate any documented
  disabilities.  If you are eligible for accommodation please contact
  me early in the semester with proper documentation from Disability
  Services.  Contact Disability Services (542-8710) about requesting
  accommodations.


\item[Mental Health and Wellness Resources]
  \begin{itemize}
  \item If you or someone you know needs assistance, you are
    encouraged to contact Student Care and Outreach in the Division of
    Student Affairs at 706-542-7774 or visit
    \url{https://sco.uga.edu}. They will help you navigate any
    difficult circumstances you may be facing by connecting you with
    the appropriate resources or services.
  \item UGA has several resources for a student seeking mental health
    services
    (\url{https://caps.uga.edu/well-being-prevention-programs-mental-health/})
    or crisis support
    (\url{https://healthcenter.uga.edu/emergencies/}).
  \item If you need help managing stress anxiety, relationships, etc.,
    please visit BeWellUGA
    (\url{https://caps.uga.edu/well-being-prevention-programs-mental-health/})
    for a list of FREE workshops, classes, mentoring, and health
    coaching led by licensed clinicians and health educators in the
    University Health Center.
  \item Additional resources can be accessed through the UGA App. 
  \end{itemize}

\end{description}

\clearpage
\begin{longtable}{ |c|c|p{10em}|p{10em}|p{10em}| }
  \hline
  \textbf{Week} & \textbf{Date} & \textbf{Topics} & \textbf{Assignment Date} & \textbf{Assignment Due} \\
  \hline\ 
  1 & 08/14 & {Introduction to GRSC 7770 Discussion: Declaration of values} 
            & {Reading assignment 1:  pages $1--6$}
            & {Sign up for CDI training} \\  \hline
  
  2 & 08/19 &  {Discussion: expectations, goals}
            &
            & {Sign up for presentations}  \\ \hline
  
    & 08/21 & {Fostering student engagement}
            & {Draw your classroom}
            & \\ \hline

  3 & 08/26 & {Presentations}
            & {} 
            & \\ \hline

    & 08/28 &  {Presentations}
            & & {} \\ \hline

  4 & 09/02 & {Presentations} 
            & {} 
            & {}\\ \hline

    & 09/04 &  {Presentations} & {} & {Proof of sign up for CDI training.}\\ \hline


  5 & 09/09 & {Fipped Classroom Discussion}
            & {Reading assignment 2: pages $7--22$} 
            & {} \\ \hline 
  
    & 09/11 & {Planning for class time. How to ask questions.} 
            & {}
            & {} \\ \hline

  6 & 09/16 &  {Guest: Philip Griffeth Academic Honesty} 
            & \multicolumn{2}{c|}{\url{https://honesty.uga.edu/AcademicIntegrityModules/}}
            \\ \hline
            % Phillip C Griffeth <pgriff@uga.edu>, Courtney Sarah Cullen <court13@uga.edu>, Peggy Jarman <PEGGY.JARMAN@uga.edu>

    & 09/18 &  {Precalculus Course Calendar}
            & {Sign up for observations.} & {} \\ \hline

  7 & 09/23 & {Recap Presentations} 
            & {} 
            & {} \\ \hline

    & 09/25 &  {Recap Presentations} & {} & {Sign up for observations.} \\ \hline

  8 & 09/30 & {Recap Presentations}
            & {} 
            & {} \\ \hline

    & 10/02 & {Recap Presentations} 
            & {} & \\ \hline

  9 & 10/07  & {Classroom Practices}
             &
             & {} \\ \hline

     & 10/09 & {Classroom Practices}
             & {}
             & {} \\ \hline

  10 & 10/14 & {Classroom Practices} 
             & {Write a quiz.}
             & {} \\ \hline

     & 10/16 & {Classroom Practices}
             & {}
             & {} \\ \hline

  11 & 10/21  & {Kelly Farmer} % Kelly L Farmer <kfarm@uga.edu>
              & {CTL Resources} 
              & {} \\ \hline

     & 10/23  & {Assessments and Grading}
              & {Write a rubric.}
              & {Observation analysis.} \\ \hline

  12 & 10/28 & {Assessments and Grading} 
             & {}  
             & {Quizzes} \\ \hline

     & 10/30 & {Assessments and Grading}
             & {Write a syllabus.}
             & {} \\ \hline

  13 & 11/04 & {Assessments and Grading} 
             & 
             & {} \\ \hline % 

     & 11/06 & {Syllabus} 
             & {Map out first day of class}
             & {Syllabus} \\ \hline % 

  14 & 11/11 & \textit{Guest: Kelly Jones , Library Resources}  % https://www.libs.uga.edu/instruction/request   kelly.jones25@uga.edu 
             & 
             & {} \\ \hline % dhartle@uga.edu


     & 11/18 & {First Day of Class}
             & 
             & \\ \hline

     
  15 & 11/20 & {Guest: Alejandra Ferguson - Equity and Inclusion} % lrcarter@uga.edu or Alejandra Ferguson alejandra.ferguson@uga.edu 
             & 
             & \\ \hline

     & 11/20 & \textit{Syllabus and goals}
             & 
             & \\ \hline

  16 & 11/25 &  {}   
             & 
             & \\ \hline
             
     & 11/27 &  {Thanksgiving Break}   
             & 
             & \\ \hline

  17 & 12/02 &  {Discussion Meet and greet with MATH 9005}   
             & 
             & \\ \hline
     & 12/04 &  {}   
             & 
             & \\ \hline

             
\end{longtable}

%%% Local Variables:
%%% mode: latex
%%% TeX-master: t
%%% End:


\end{document} 
              
%%% Local Variables:
%%% mode: latex
%%% TeX-master: t
%%% End:
